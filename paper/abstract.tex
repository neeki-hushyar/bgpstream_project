 \section{Abstract}\label{sec:abstract}
Broader Gateway Protocol (BGP) is an insecure protocol. According to BGPMon project twitter-feed, BGP prefix hijacks occur at a frequency of every few hours. A major indication of a BGP prefix hijack is multiple ASes announcing the same or more-specific bgp prefix but multiple AS origins does not necessarily imply a hijack. Part of the difficulty in identifying hijack as a third party is being able to filter out non-anomalous BGP origin conflicts.\\
We present an algorithm for filtering out most of the non-malicious AS origin conflicts. We automate filtering based on CDN, peering relationships, AS organization names, country registered for the AS and duration of the hijack to get a list of events that indicate whether misconfiguration or a hijack. To verify hijack, we manually check BGP-related forums and also check the BGPStream twitter-feed via the twitter API. We explain the reasons for our algorithm to have larger number of false positives and fewer false negatives. \\
From our study, we find that a very large fraction of AS origin conflict involve same country, same organization, ASes with peering relation or some form of traffic engineering. Only a few AS origin conflict incidents look malicious. During our period of study, that is April 2017, the median duration of the hijack or misconfigurarion lasted from less than \textbf{30 }mins. Most of the hijack or misconfigurations were propagated by the USA, Vietnam, Australia, India and Bangladesh. We also find that internet provider destinations are the most common hijack destinations and most of the hijacks are initiated my smaller autonomous systems. This matches our expectation formed according to the valley-free routing. Preference given to customer paths implies that provider destinations are easy to hijack and stubs and  customers low in the hierarchy can easily initiate a hijack .