 \section{Related Work}\label{sec:relatedwork}

In 2001, researchers conducted an experiment which identified valid and invalid BGP conflicts over 3 years. Valid conflicts were due to changes in providers and multi-homing, while invalid conflicts were a result of accidental and intentional misconfigurations \cite{colorado}. To identify whether a conflict was valid or invalid, they analyzed the duration of the conflicting announcements. The theory was that the longer the conflict lasted, the higher the chance of it being invalid. Our work, as well as those mentioned in the remainder of this section, found that monitoring duration is not sufficient to identify the cause of a conflict. However, we do use the duration to rule out possible causes for a conflict - namely, traffic engineering.

Another study, focused on the duration and relationship of conflicts over 10 years \cite{euro}. This allowed for correlation between seemingly independent conflicts. They found conflicts previously labeled as resulting from multi-homing, actually resulted from an increase in ASes used for improved connectivity. Further, the correlation revealed that most misconfigurations came from the same origin ASes. This study also considered the peering relationships between ASes announcing the same prefix blocks to further identify valid conflicts. Our analysis also considers peering relationships, as well other metrics which suggest a conflict is valid.

Several monitoring systems have been proposed to stop invalid paths from propogating, in real time. The first is BGPMon which consists of a network of monitors collect data and communicate with each other to make sure announcements are legitimate \cite{bgpmon}. The monitors alert network administrators whenever an unexpected announcement is seen for some prefix. ARTEMIS is another monitor which compares any BGP update to the history of paths for the given prefix and the length of previous prefix paths \cite{artemis}. Unexpected results prompt the monitors to check reverse paths. This is similar to checking the peering relationships in our study. A weakness of monitors, is maintaining a database of the true value of the prefixes and the corresponinding administrators.

The above works aimed to identify whether conflicts were valid or invalid. While we use overlapping metrics, we probe further to characterize the specific causes of the conflicts. Furthermore, our work identifies several CDNs, which previous methods may flag as an invalid conflict.
