\section{Conclusion and Future Work}
Using BGPStream data, we searched for conflicting announcements. We successfully identified several BGP hijacks and/or accidental misconfigurations. In the process, we utilized CAIDA AS information to identify and filter out CDNs, international organizations and conflicts due to peering relationships from the conflicts we detected. A large portion of conflicts could be filtered our as same organization, as peers or originating in the same country. We found recognizable entities to be common hijack/misconfiguration destinations and entities with unfamiliar names, possibly stubs, to be the common hijackers/misconfigurers. The duration of hijacks had a median value of 23 minutes. We discuss the limitations of our algorithm in identifying prefix hijacks occurring in the same country and hijacks targeting a CDN. 

We note that our ability to label accidental misconfigurations is only as strong as the accuracy of the CAIDA relationships and AS data. We collect  our BGP announcements from BGPStream which allows us to gather data from various collectors including Route Views, RIPE and OpenBMP. While we have a full view of the prefixes announced by these collectors, we miss malicious activity by ASes not tracked by these collectors, or only partially tracked by these collectors. In addition, because of the changing internet topology, the CAIDA relationships and map dataset is always changing. This makes analyzing historical dataset on with a single threaded program on single computer hard. 

In process of writing python code for an algorithm that identifies historical hijacks as well as hijacks in real time, we have realized that the code needs to be very carefully designed. This is specially true if we are writing a single threaded code to be run on a single computer. Getting data from BGPStream is blocking and involves data transfer over a network. Similarly, our current code holds AS to organization mapping and memory information in memory. Given the size of the data we are looking at, this slows down our computation a lot. Running our algorithm for our data set, an hour of data for the month of April in 2017, easily takes more than 10 minutes. Time sensitive nature of identifying BGP hijacks demands that our program run fast. Therefore, an obvious future direction will be to make the code multi-threaded. Given the parallel nature of our workflow, using distributed computation in the form of a map-reduce architecture will speed up the program even more. Another obvious extension after adding distributed computing capabilities would be to use machine learning for classifying hijacks.

In addition, future studies can include following misconfigurations over time and correlating the results to the similar data from known hijacks. This would allow us to identify several hijacks that we currently classified as misconfigurations. 
