\section{Discussion}\label{sec:discussion}
Our findings matched our expectations. The only surprising \textbf{result was }
\subsection{Algorithm Validation}
Our algorithm successfully identified all three of the test hijacks. Thinking backwards, going in with the intention of identify the three hijacks helped us fine-tune our algorithm. In each of the three cases, our algorithm identified the intended hijacks and labeled some more incidents as `hijacks or misconfigurations'. This lead us to believe that our algorithm has a higher rate of false positives and lower rate of false negatives.\\
A major and an obvious source of false positives is the inability to automate separating malicious hijacks from accidental misconfigurations. Solutions to this would be using machine learning or mining bgp-related forum data which can be very expensive computationally. One of the limitations of our algorithm is the inability to identify malicious hijacks originating in the same country. Our algorithm filters out origin ASes in the same country as being non-threatening. Second, our algorithm filters out CDNS which means attacks target CDNs are not identified. These would be our sources of our false negatives. 
\subsection{Types of origin AS conflicts}
We found that most of the prefix origin conflicts result within the same country. This is not surprising given distribution of number of ASes by country is very asymmetric with large number of ASes belonging to countries such as Germany and the United States. As discussed previously, our algorithm calculates proportion of conflicting prefixes origins in the same country exclusive of CDNs and ASes belonging to the same organizations. Hence, the total number of conflicting ASes will be even higher \textbf{than}. 
We were expecting the proportion of CDNs to be a little higher due to the fact that we look for CDNs as the first step in our algorithm. This could be a limitation of the list of CDNs that we identify. However, we do identify twenty most popular CDNs which means the proportion of organizations owning multiple ASes is much much higher than \textbf{proportion} of organizations using CDNs. This indicates the existing potential market for CDNs. There is trend of moving content delivery towards the of the network and away from the backbone. Organizations owning multiple ASes can potentially save costs by moving non-sensitive contents to CDNs.
\subsection{Hijack Duration  and Location}
The median hijack duration\textbf{ of is} reasonable given today's BGP monitoring infrastructure. We believe most hijacks are detected within a few minutes and rectified. Unlike the Pakistan hijack, during our period of study, we did not see hijacks or misconfigurations being mitigated by announcing a more specific prefix. The prefix lengths announced by the origin and the attacker \textbf{remained the same in most cases}. This leads us to believe that most of the incidents we detected were misconfigurations rather than hijacks and they were detected and mitigated very fast.
\subsection{Common destinations and attackers}
\textbf{are} some of the organizations who were victims of the hijack during our study period. In contrast, non-providers were the found as the ASes that caused the hijack. This reinforces the idea that provider destinations are more easily hijacked. This can be explained by the valley-free routing assumption. Provider paths tend to be expensive, therefore path via a customer is preferred. Any customer announcing a provider path would win if origin authentication is not in place.
\subsection{Prevention}
 There are several suggested methods for BGP hijacking prevention. The simplest prevention method would be to only allow updates from the autonomous system which previously announced the path for a specific entity. This, however may negatively affect the availability and reachability of the end system. A compromise which increases the security of the update while maintaining availability, would be to delay the adoption of an update which originates from an unexpected AS. In order for this method to be implemented, the AS and AS-level edges which normally announce each BGP path, or blocks of paths, may need to be white listed in some sense. BGP updates would have to be monitored and processed. This mechanism allows time for an accidental update be removed before having widespread, if any affect. 
 
An inactive implementation of such system, is the Internet Alert Registry (IAR). The IAR  maintained routing security information based on route history. Route history determined BGP prefix ownership and an alert was raised if an update pointed back to a different prefix. A danger of this implementation is that a trusted AS-level edge for a given prefix will not be flagged in the case it makes a false announcement. Further, in the case of the Pakistan hijack, the falsely announced route would not have been flagged as the specific prefix announced would have still been in the valid prefix space. The following mechanism would catch the previously mentioned updates which would avoid detection by the IAR.

The mechanism proposed in [...cite], creates a system to flag and alert prefix owners to any update which changes the origin. Further, sending alerts for every update has high overhead on both the monitor and on the BGP prefix owner.  

All the above prevention and detection mechanisms rely on data pooled from monitors, route server and registries. This data is not comprehensive of all BGP routes and prefixes on the internet and can only prevent/detect attacks which it can map a route for.
 
 